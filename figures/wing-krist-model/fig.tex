\documentclass[,dvipsnames]{article}
\usepackage{geometry}
\geometry{
    papersize={190mm,100mm},
    left=0mm,
    right=0mm,
    top=0mm,
    bottom=0mm,
    headheight=0mm,
}
\usepackage{fontspec}
\setmainfont{Noto Sans}

\usepackage{tikz}
\usetikzlibrary {arrows.meta} 
\usetikzlibrary {positioning}

%%% standard math packages for equations:
\usepackage{amsmath}
\usepackage{amssymb}
\usepackage{mathtools}

\usepackage{annotate-equations}
\usepackage{pgfmath}
\usepackage{ifthen}

\DeclareMathOperator{\E}{\mathbb{E}}
\definecolor{Gray}{HTML}{BBBBBB}
\definecolor{LightBlue}{HTML}{66CCEE}
\definecolor{Blue}{HTML}{4070A0}
\definecolor{BrightPink}{HTML}{B33644}
\definecolor{Magenta}{HTML}{AA3377}
\definecolor{Gold}{HTML}{CCBB44}
\definecolor{Green}{HTML}{228833}

\begin{document}

\begin{figure}[ht]
% \centering
\advance\leftskip2.5em
\begin{tikzpicture}[scale=1.5]
    \fill[Gray!25]      (-4.1,.5em) rectangle +(11,-3.05em); % Unobserved area
    \fill[LightBlue!20] (-4.1,-2.55em) rectangle +(11,-3.3em); % Observed area

    \draw (-3.05,-0.6em) node[anchor=north east] {\scshape Trigger};
    \draw (-2.75,-2.62em) node[anchor=north east] {\scshape Response};
    
    \draw (6.9,-2.55em) node[anchor=south east] {\scshape Unobserved};
    \draw (6.9,-5.85em) node[anchor=south east] {\scshape Observed};

    \draw[very thick, arrows={-{Stealth[round]}}] (-4.2,-.5em) -- +(11.3,0) node[right=0.25em] {\scshape Time};

    \foreach \c/\d/\cj/\dj/\l/\cjr/\djr in {-3/0.3/-1/-2/3/0/1, 0.0/0.7//-1/3.9/0/0, 3.9/0.5/+1//2.8/1/0} {
        % \settowidth{\w}{$C_{j\j}$}
        % \pgfmathparse{\w/2}
        \node (C) at (\c cm + \l cm/2,0) [rectangle] {$C_{j\cj}$};
        \draw[arrows={{Straight Barb[round,width=3pt]}-}] (\c+.05,0em) -- (C);
        \ifthenelse{\cjr=1}
            {\draw[arrows={{Circle[sep,length=1pt] . Circle[sep,length=1pt] .Circle[sep,length=1pt] . Circle[sep,length=0pt]}-}] (\c+\l-.05,0em) -- (C);}
            {\draw[arrows={{Straight Barb[round,width=3pt]}-}] (\c+\l-.05,0em) -- (C);}
        ;

        \draw[ultra thick, arrows={-{Stealth[round]}}] (\c,0.1) -- +(0,-1.8em); % clock trigger instance arrow

        \draw[ultra thick, arrows={-{Stealth[round]}}] (\c+\d,0.1-2.0em) -- +(0,-1.5em); % motor delay response arrow

        % Motor delay responses
        \draw                                                  (\c,       0.1-2.4em) -- +(0,.4em); % vertical bound line
        \draw[arrows={-{Straight Barb[length=2pt,width=2pt]}}] (\c-.1,    0.1-2.2em) -- +(.1,0); % arrows
        \draw[arrows={{Straight Barb[length=2pt,width=2pt]}-}] (\c+\d+.02,0.1-2.2em) -- +(.1,0); % arrows

        \ifthenelse{\djr = 1}
            {\draw (\c+\d+.02, 0.1-2.25em) node[right=0.25em] {$\scriptstyle D_{j\dj}$};}
            {\draw (\c+\d/2, 0.1-2.25em) node {$\scriptstyle D_{j\dj}$};}
        ;

        \draw (\c,0.1-1.6em) -- +(\d,-.3em); % diagonal cue to delay line
        \draw (\c+\d,0.1-4.2em) -- +(0,.5em); % IRI vertical bound line
    }

    \node (Ij1) at (-1.1,0.1-3.95em) [rectangle] {$I_{j-1}$};
    \draw[arrows={-{Straight Barb[round,width=3pt]}}] (Ij1) -- +(-1.6+.05,0);
    \draw[arrows={-{Straight Barb[round,width=3pt]}}] (Ij1) -- +(1.8-.05,0);

    \setlength{\abovedisplayskip}{0pt}
    \setlength{\belowdisplayskip}{0pt}

    \node [rectangle, text width=3cm,  above right=-4.84ex and 3.25cm of Ij1.north east] (eq_ij) {
        \renewcommand{\eqnhighlightheight}{}
        \renewcommand{\eqnhighlightshade}{35}
        \renewcommand{\eqnannotateshade}{100}
        \begin{equation*}
            I_{j} = 
            \eqnmarkbox[Blue]{c}{C_{j}} +
            \eqnmarkbox[Magenta]{d1}{D_{j}} -
            \eqnmarkbox[Magenta]{d2}{D_{j-1}}
        \end{equation*}
        \annotate[yshift=-.71em]{below,left}{c}{Internal clock interval}
        \annotatetwo[yshift=-.75em]{below, label below}{d1}{d2}{Motor delays}
    };
    \coordinate[right=0em of eq_ij.west] (eq_ij_left);
    \coordinate[left=-1.7em of eq_ij.east] (eq_ij_right);
    \draw[arrows={{Straight Barb[round,width=3pt]}-}] ++(0.7+.05,0.1-3.95em) coordinate(a) -- (a -| eq_ij_left);
    \draw[arrows={{Straight Barb[round,width=3pt]}-}] ++(4.4-.05,0.1-3.95em) coordinate(a) -- (a -| eq_ij_right);

    \renewcommand{\eqnhighlightshade}{19}
    \draw (-4.0,-2.5) node [rectangle, text width=5cm, anchor=north west] {
        \renewcommand{\eqnhighlightheight}{}
        \renewcommand{\eqnannotateshade}{100}
        \begin{flalign}
        \eqnmarkbox[Blue]{p1}{\rho_{I}(1)} &= 
        \frac{
            \eqnmarkbox[BrightPink]{g1}{\gamma_{I}(1)}
        }{
            \eqnmarkbox[Magenta]{g0}{\gamma_{I}(0)}
        }
        \\[6.5ex]
        \sigma_{D}^{2} &= -\gamma_{I}(1) \tag{4}
        \end{flalign}
        \annotate[yshift=0.75em]{above,left}{p1}{lag(1) correlation}
        \annotate[yshift=.75em]{above}{g1}{lag(1) covariance}
        \annotate[yshift=-.5em]{below}{g0}{lag(0) covariance}
    };

    \draw (0.75,-2.5) node [rectangle, text width=8cm, anchor=north west] {
        \renewcommand{\eqnhighlightheight}{}
        \renewcommand{\eqnannotateshade}{105}
        \begin{flalign}
            \gamma_{I}(1) &= 
                \eqnmarkbox[Magenta]{E}{\!\E\!}\!\left[ (I_{j} - 
                \eqnmarkbox[Blue]{mu1}{\mu_{I}}
                )(I_{j-1} - 
                \eqnmarkbox[Blue]{mu2}{\mu_{I}}
                )  \right] = -
                \eqnmarkbox[Green]{varD}{\sigma_{D}^{2}} \\
            \gamma_{I}(0) &= \E\! \left[ \left(I_{j} - \mu_{I}\right)\left(I_{j} - \mu_{I}\right)  \right] = \sigma_{I}^{2} \\
                &= 
                \eqnmarkbox[BrightPink]{varC}{\sigma_{C}^{2}}
                + 2\sigma_{D}^{2} \nonumber
            \\[2.75ex]
            \sigma_{C}^{2} &= \gamma_{I}(0) + 2\gamma_{I}(1) \tag{5}
        \end{flalign}
        \annotate[yshift=0.75em]{above,left}{E}{Expected value}
        \annotatetwo[yshift=0.75em]{above,right}{mu1}{mu2}{Average response interval}
        \annotate[yshift=0.75em]{right}{varD}{Motor delay variance}
        \annotate[yshift=0.25em,xshift=4.0em]{below,right}{varC}{Internal clock variance}
    };

\end{tikzpicture}
\vspace{-3ex}
\caption{
    \footnotesize
    Schematic and model for the Wing-Kristofferson model of the timing of repetitive, discrete motor responses. It is assumed that an internal clock process generates trigger pulses at intervals $C_j$. A response occurs an independent delay, $D_j$,  after each trigger to account for e.g. neuromuscular transmission time, movement time, etc. Neither $C_j$ or $D_j$ are directly observable from response timing, however, the clock and motor variance, $\sigma_C^2$ and $\sigma_D^2$, can be estimated from the
    characteristics of the interresponse intervals $I_j$. The lag one serial correlation $\rho_I(1)$, which measures dependence of consecutive interresponse intervals, can be negative without explicit negative feedback in the system. For example, a zero variance internal clock (e.g. if perfectly following an external metronome cue, and generally when $\sigma_D^2 \gg \sigma_C^2$) would lead to $\rho_I(1)=-1/2$. See Wing and Kristofferson (1973) for derivations and further
    details.
}
\end{figure}

\end{document}

